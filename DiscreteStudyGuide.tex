\documentclass[12pt]{article}

\usepackage{answers}
\usepackage{setspace}
\usepackage{graphicx}
\usepackage{enumitem}
\usepackage{multicol}
\usepackage{mathrsfs}
\usepackage[margin=1in]{geometry} 
\usepackage{makecell}
\usepackage{amsmath,amsthm,amssymb}
 
\newcommand{\N}{\mathbb{N}}
\newcommand{\Z}{\mathbb{Z}}
\newcommand{\C}{\mathbb{C}}
\newcommand{\R}{\mathbb{R}}

\DeclareMathOperator{\sech}{sech}
\DeclareMathOperator{\csch}{csch}
\newcommand{\powerset}{\raisebox{.15\baselineskip}{\Large\ensuremath{\wp}}}
 
\newenvironment{theorem}[2][Theorem]{\begin{trivlist}
\item[\hskip \labelsep {\bfseries #1}\hskip \labelsep {\bfseries #2.}]}{\end{trivlist}}
\newenvironment{definition}[2][Definition]{\begin{trivlist}
\item[\hskip \labelsep {\bfseries #1}\hskip \labelsep {\bfseries #2.}]}{\end{trivlist}}
\newenvironment{proposition}[2][Proposition]{\begin{trivlist}
\item[\hskip \labelsep {\bfseries #1}\hskip \labelsep {\bfseries #2.}]}{\end{trivlist}}
\newenvironment{lemma}[2][Lemma]{\begin{trivlist}
\item[\hskip \labelsep {\bfseries #1}\hskip \labelsep {\bfseries #2.}]}{\end{trivlist}}
\newenvironment{exercise}[2][Exercise]{\begin{trivlist}
\item[\hskip \labelsep {\bfseries #1}\hskip \labelsep {\bfseries #2.}]}{\end{trivlist}}
\newenvironment{solution}[2][Solution]{\begin{trivlist}
\item[\hskip \labelsep {\bfseries #1}]}{\end{trivlist}}
\newenvironment{problem}[2][Problem]{\begin{trivlist}
\item[\hskip \labelsep {\bfseries #1}\hskip \labelsep {\bfseries #2.}]}{\end{trivlist}}
\newenvironment{question}[2][Question]{\begin{trivlist}
\item[\hskip \labelsep {\bfseries #1}\hskip \labelsep {\bfseries #2.}]}{\end{trivlist}}
\newenvironment{corollary}[2][Corollary]{\begin{trivlist}
\item[\hskip \labelsep {\bfseries #1}\hskip \labelsep {\bfseries #2.}]}{\end{trivlist}}
 
\begin{document}
 
% --------------------------------------------------------------
%                         Start here
% --------------------------------------------------------------
 
\title{FINAL Study Guide}%replace with the appropriate homework number
\author{Andrew Reed\\ %replace with your name
Discrete Final} %if necessary, replace with your course title
 
\maketitle

\tableofcontents

\pagebreak

\quad

\quad

\section{The Foundations: Logic and Proofs}

\subsection{Propositional Logic}

Examples of a proposition:

$p(x) = x$ is a cat.

$q(x) = x$ has fur.

\begin{enumerate}
\item \underline{Negation}, $\neg p(x)$, changing the statement to $x$ is not a cat
\item \underline{Conjunction}, $p(x) \wedge q(x)$, changing the statement to say $x$ is a cat and it has fur.
\item \underline{Disjunction}, $p(x) \lor q(x)$, changing the statement to say $x$ is a cat and it does not have fur.
\item \underline{Exclusive Or},  $p(x) \oplus q(x)$, where the statement is true only when exactly one of $p(x)$ or $q(x)$ is true, otherwise the statement is false.
\item \underline{Conditional Statement}, $p(x) \rightarrow q(x)$,  where the statement is read "If p, then q" the statement is false when p is true and q is false, otherwise the statement is true.
\item \underline{Bi-Conditional Statement}, $p(x) \leftrightarrow q(x)$, where the statement is true if the values of p and q match. The statement is read as any of the following;
	\begin{enumerate}
	\item "p if and only if q"
	\item "p is necessary and sufficient for q"
	\item "if p then q, and conversely"
	\item "p iff q"
	\end{enumerate}
\item \underline{Converse, Contrapositive, and Inverse}

	Given $p \rightarrow q$;
	\begin{enumerate}
	\item Converse: $q \rightarrow p$
	\item Contrapositive: $\neg q \rightarrow \neg p$
	\item Inverse: $\neg p \rightarrow \neg q$
	\end{enumerate}
\end{enumerate}

Definitions
\begin{enumerate}
\item \underline{Hypothesis} - p is considered the hypothesis in the statement $p(x) \rightarrow q(x)$.
\item \underline{Conclusion}- q is considered the conclusion in the statement $p(x) \rightarrow q(x)$.
\item \underline{Bi-Implications} - Another way to express "Bi-Conditional statements"
\end{enumerate}

\quad

\quad

\subsection{Propositional Equivalences}

\begin{enumerate}
\item \underline{Logically Equivalent}, $p(x) \equiv q(x)$, when two statements share the same truth values then the statement is said to be logically equivalent.
\item \underline{Identity Laws} 
\[
	p(x) \wedge T \equiv p(x)
\]
\[
	p(x) \lor F \equiv p(x)
\]

\item \underline{Domination Laws}
\[
	p(x) \lor T \equiv T 
\]
\[
	p(x) \wedge F \equiv F
\]

\item \underline{Idempotent Laws}
\[
	p(x) \lor p(x) \equiv p(x)
\]
\[
	p(x) \wedge p(x) \equiv p(x)
\]

\item \underline{Double Negation Law}
\[
	\neg (\neg p(x)) \equiv p(x)
\]

\item \underline{Commutative Laws}
\[
	p(x) \lor q(x) \equiv q(x) \lor p(x) 
\]
\[
	p(x) \wedge q(x) \equiv q(x) \wedge p(x)
\]

\item \underline{Associative Laws}
\[
	\left ( p(x) \lor q(x) \right ) \lor r(x) \equiv  p(x) \lor \left ( q(x)  \lor r(x) \right )
\]
\[
	\left ( p(x) \wedge q(x) \right ) \wedge r(x) \equiv  p(x) \wedge \left ( q(x)  \wedge r(x) \right )
\]

\item \underline{Distributive Laws}
\[
	p(x) \lor \left ( q(x)  \wedge r(x) \right ) \equiv \left ( p(x) \lor q(x) \right ) \wedge \left ( p(x) \lor r(x) \right )
\]
\[
	p(x) \wedge \left ( q(x)  \lor r(x) \right ) \equiv \left ( p(x) \wedge q(x) \right ) \lor \left ( p(x) \wedge r(x) \right )
\]

\item \underline{De Morgan's Laws}
\[
	\neg \left ( p(x) \wedge q(x) \right ) \equiv \neg p(x) \lor \neg q(x)
\]
\[
	\neg \left ( p(x) \lor q(x) \right ) \equiv \neg p(x) \wedge \neg q(x)
\]

\item \underline{Absorption Laws}
\[
	p(x) \lor \left ( p(x) \wedge q(x) \right ) \equiv p(x)
\]
\[
	p(x) \wedge \left ( p(x) \lor q(x) \right ) \equiv p(x)
\]

\item \underline{Negation Laws}
\[
	p(x) \lor \neg p(x) \equiv T
\]
\[
	p(x) \wedge \neg p(x) \equiv F
\]

\end{enumerate}

Definitions
\begin{enumerate}
\item \underline{Tautology} - When all cases in the statement are determined to be true, it is said to be a tautology.
\item \underline{Contradiction}- When all cases in the statement are determined to be false, it is said to be a contradiction.
\item \underline{Contingency} - When the statement is neither a tautology, nor a contradiction then it is said to be a contingency.
\end{enumerate}

\quad

\quad

\subsection{Predicates and Quantifiers}

\quad 

\begin{enumerate}
\item \underline{Universal Quantification}, $\forall p(x)$ read as, "For all $x$, $p(x)$."
\item \underline{Existential Quantification}, $\exists p(x)$ read as, "There exists an element $x$, that $p(x)$."
\item \underline{De Morgan's Laws for Quantifiers}:
\[
	\neg \forall p(x) \equiv \exists \neg p(x)
\]
\[
	\neg \exists p(x) \equiv \forall \neg p(x)
\]
\end{enumerate}

Definitions
\begin{enumerate}
\item \underline{Quantification} - Used to express the extent that a predicate is true. In English, the words; all, some, many, none, and few are used in quantifications.
\item \underline{Counterexample} - When there is a value, $x$ in which $\forall p(x)$ is false, then that value of $x$ is called a counterexample of $\forall p(x)$.
\end{enumerate}

\pagebreak

\subsection{Rules of Inference (ROI)}

\quad 

\begin{center}
\begin{tabular}{ c c }

\makecell{ \underline{Modus Ponens} \\ p \\ p $\rightarrow$ q \\ $\overline{\therefore{q}}$ \\ \quad}

&

\makecell{ \underline{Modus Tollens} \\ $\neg q$ \\ p $\rightarrow$ q \\ $\overline{\therefore{\neg p}}$ \\ \quad}

\\

\makecell{ \underline{Hypothetical Syllogism} \\ p $\rightarrow$ q \\ q $\rightarrow$ r \\ $\overline{\therefore{p \rightarrow r}}$ \\ \quad}

&

\makecell{ \underline{Disjunctive Syllogism} \\ p $\lor$ q \\ $\neg$ p \\ $\overline{\therefore{q}}$ \\ \quad}

\\

\makecell{ \underline{Addition} \\ p \\ $\overline{\therefore{p \lor q}}$ \\ \quad}

&

\makecell{ \underline{Simplification} \\ p $\wedge$ q \\ $\overline{\therefore{p}}$ \\ \quad}

\\

\makecell{ \underline{Conjunction} \\ p \\ q \\ $\overline{\therefore{p \wedge q}}$ \\ \quad }

&

\makecell{ \underline{Universal Instantiation} \\ $\forall p(x)$ \\ $\overline{\therefore{p(c)}}$ \\ \quad }

\\

\makecell{ \underline{Universal Generalization} \\ $p(c)$ for an arbitrary c \\ $\overline{\therefore{\forall p(x)}}$ \\ \quad}

&

\makecell{ \underline{Existential Instantiation} \\ $\exists p(x)$ \\ $\overline{\therefore{p(c)}}$ for some element c \\ \quad}

   
\end{tabular}

\quad

\quad

\underline{Existential Generalization}

$p(c)$ for some element c

 $\overline{\therefore{\exists p(x)}}$ 

\end{center}


Definitions
\begin{enumerate}
\item \underline{Argument} - Sequence of propositions.
\item \underline{Premises} - All propositions in the argument with the exclusion of the conclusion
\item \underline{Conclusion} - The final proposition in the argument.
\item \underline{Argument Form} - A sequence of compound propositions involving propositional variables.
\item \underline{Valid} - A form that makes it impossible for the premises to be true and the conclusion nevertheless to be false.
\end{enumerate}

\subsection{Proofs}


\begin{enumerate}
\item \underline{direct proof}:

Directly prove that if n is an odd integer then $n^{2}$ is also an odd integer.

PROOF ($\rightarrow$): 

An odd number is denoted by the equation, $2k + 1$

\[
	n^{2} = (2k + 1)^{2} = 4k^{2} + 4k + 1 = 2(2k^{2} + 2) + 1
\]
 $2(2k^{2} + 2) + 1$, is another form of 2k + 1.
 
 Therefore, if n is an odd integer then $n^{2}$ is also an odd integer.
 
 \item \underline{Proofs by Contradiction}:
 
 
\end{enumerate}

\quad

\quad

Definitions
\begin{enumerate}
\item \underline{Theorem} - A statement that can be shown to be true.
\item \underline{Proof} - A demonstration that a theorem is true.
\item \underline{Axioms} (or postulates) - Statements we assume to be true.
\item \underline{Lemma} - A less important theorem that is helpful in the proof of other results.
\item \underline{Corollary} - A theorem that can be established directly from a theorem that has been proved.
\item \underline{Conjecture} - A statement that is being proposed to be a true statement, usually on the basis of some partial evidence, a heuristic argument, or the intuition of an expert. When a proof of a conjecture is found, the conjecture becomes a theorem. 
\end{enumerate}

\pagebreak

\section{Sets, Functions, Sequences, Sums, and Matrices}

\quad

\quad

\subsection{Sets}

\quad

\noindent A set is an unordered collection of objects, called elements or members of the set. 

We write $a \in A$ to denote that a is an element of the set A. 

We write $a \notin A$ denotes that a is not an element of the set A

\quad

\noindent Well known sets include;

$\N=\{ 0, 1, 2, 3, . . . \}$, the set of natural numbers

$\Z = \{ . . . , -2, -1 ,0 ,1 ,2 , . . . \}$, the set of integers

$\Z += \{1, 2, 3, . . . \}$, the set of positive integers

$Q = \{p/q | p \in \Z,\ q \in \Z,\ and\ q \neq  0\}$, the set of rational numbers R, the set of 

\indent \indent real numbers

$\R+$, the set of positive real numbers

$\C$, the set of complex numbers.

$\Omega$, the universal set

\quad

\noindent \underline{Showing that A is a Subset of B} To show that $A \subseteq B$, show that if x belongs to A then x 
\indent also belongs to B.

\quad

\noindent \underline{Showing that A is Not a Subset of B} To show that $A \nsubseteq B$, find a single x $\in$ A such that  

\indent x $\notin$ B.

\quad

\noindent \underline{Showing Two Sets are Equal} To show that two sets A and B are equal, show that $A \subseteq B$ and $B \subseteq A$.

\quad

\noindent Every set (S) includes the following two sets;

\[
	S \subseteq S \quad \quad AND \quad \quad \emptyset \subseteq S
\]

\quad

\noindent A set (S) is said to be a \underline{finite set} when it has exactly n distinct elements and n is said to be \indent the \underline{cardinality} of S, denoted as $|S|$

\quad

\noindent The \underline{Power Set} of a set (S) is the set of all subsets of S. It is denoted as $\powerset (S)$

\quad

\quad

\subsection{Set Operations}

\quad

\noindent Let A and B be sets. The \underline{Cartesian product} of A and B, denoted as $ A \times B $, is the set of all \indent ordered pairs $(a, b)$ where $ a \in A$ and $b \in B$

\quad

\noindent Let A and B be sets. The \underline{union} of the sets A and B, denoted by $A \cup B$, is the set that \indent contains those elements that are either in A or in B, or in both.

\quad

\noindent Let A and B be sets. The \underline{intersection} of the sets A and B, denoted by A $ \cap $ B, is the set \indent containing those elements in both A and B.

\quad

\noindent Two sets are called \underline{disjoint} if their intersection is the empty set.

\quad

\noindent Let A and B be sets. The \underline{difference (complement)} of A and B, denoted by $A - B$, is the set \indent containing those elements that are in A but not in B.

\quad

\noindent The \underline{complement} of the set A, denoted by $\overline{\rm A}$, is the complement of A with respect to $\Omega$ . \indent Therefore, the complement of the set A is $\Omega$ - A.

\quad

\pagebreak

\subsection{Set Identities}

\begin{enumerate}
\item \underline{Identity Laws}
\[
	A \cap \Omega = A
\]
\[
	A \cup \emptyset = A
\]
\item \underline{Domination Laws}

\[
	A \cup \Omega = \Omega
\]
\[
	A \cap \emptyset = \emptyset
\]
\item \underline{Idempotent Laws}
\[
	A \cup A = A
\]
\[
	A \cap A = A
\]
\item \underline{Complementation Laws}
\[
	\overline{\left ( \overline{\rm A} \right ) } = A
\]
\item \underline{Commutative Laws}
\[
	A \cup B = B \cup A
\]
\[
	A \cap B = B \cap A
\]
\item \underline{Associative Laws}
\[
	A \cup \left ( B \cup C \right ) =  \left ( A \cup B \right ) \cup C 
\]
\[
	A \cap \left ( B \cap C \right ) =  \left ( A \cap B \right ) \cap C 
\]
\item \underline{Distributive Laws}
\[
	A \cup \left ( B \cap C \right ) =  \left ( A \cup B \right ) \cap \left ( A \cup C \right )
\]
\[
	A \cap \left ( B \cup C \right ) =  \left ( A \cap B \right ) \cup \left ( A \cap C \right )
\]
\item \underline{De Morgan's Laws}
\[
	\overline{\rm A \cup B} = \overline{\rm A} \cap \overline{\rm B}
\]
\[
	\overline{\rm A \cap B} = \overline{\rm A} \cup \overline{\rm B}
\]
\item \underline{Absorption Laws}
\[
	A \cup \left ( A \cap B \right ) = A
\]
\[
	A \cap \left ( A \cup B \right ) = A
\]
\item \underline{Complement Laws}
\[
	A \cup \overline{\rm A} = \Omega
\]
\[
	A \cap \overline{\rm A} = \emptyset
\]
\end{enumerate}

\subsection{Functions}

Given A function $f$ from A to B
\begin{enumerate}

\item \underline{One To One}: For a function to be one to one or every element $b \in B$ there is an element $a \in A$ with $f(a)=b$.

\item \underline{Onto}: For a function to be onto $f(a) = f (b)$ implies that $a = b$ for all $a$ and $b$ in the domain of $f$.

\end{enumerate}

Definitions:
\begin{enumerate}

\item \underline{Function} - Let A and B be nonempty sets. A function $f$ from A to B is an assignment of exactly one element of B to each element of A. We write $f(a) = b$ if b is the unique element of B assigned by the function $f$ to the element a of A. If f is a function from A to B, we write f : $A \rightarrow B$.
\item \underline{Domain} - The set of possible values of the variables of a function.
\item \underline{Range} - The set of all output values of a function.
\item \underline{Surjective} - Another way to say onto.
\item \underline{Injective} - Another way to say one to one.
\item \underline{One-to-one Correspondence (Bijection)} - Both onto and one to one.
\end{enumerate}

\subsection{Sequences and Summations}
\subsection{Matrices}

\section{Induction and Recursion}

\subsection{Mathematical Induction}

\begin{enumerate}
\item \underline{Principle of Mathematical Induction} To prove that $P(n)$ is true for all positive integers $n$, where $P(n)$ is a propositional function, we complete two steps, the base step and inductive step.
\item \underline{Base Step} We verify that P (1) is true.
\item \underline{Inductive Step} We show that the conditional statement $P(k) \rightarrow P(k + 1)$ is true for all positive integers k.
\end{enumerate}

\pagebreak

Template for Proofs by Mathematical Induction

\begin{enumerate}
\item Express the statement that is to be proved in the form ?for all n ? b, P (n)? for a fixed integer b.
\item Write out the words ?BasisStep.? Then show that $P(b)$ is true, taking care that the correct value of b is used. This completes the first part of the proof.
\item Write out the words ?Inductive Step.?
\item State, and clearly identify, the inductive hypothesis, in the form ?assume that $P(k)$ is true for an arbitrary fixed integer $k \geq b$.?
\item State what needs to be proved under the assumption that the inductive hypothesis is true. That is, write out what $P(k + 1)$ says.
\item Prove the statement $P(k + 1)$ making use the assumption $P(k)$. Be sure that your proof is valid for all integers $k$ with $k \leq b$, taking care that the proof works for small values of $k$, including $k = b$.
\item Clearly identify the conclusion of the inductive step, such as by saying ?this completes the inductive step.?
\item After completing the basis step and the inductive step, state the conclusion, namely that by mathematical induction, $P(n)$ is true for all integers $n$ with $n \geq b$.
\end{enumerate}

\subsection{Strong Induction}

\begin{enumerate}
\item \underline{Strong Induction} To prove that $P(n)$ is true for all positive integers $n$, where $P(n)$ is a propositional function, we complete two steps, the base step and inductive step.
\item \underline{Base Step} We verify that P (1) is true.
\item \underline{Inductive Step} We show that the conditional statement $\left [ P (1) \wedge P(2) \wedge . . . \wedge P(k) \right ]\rightarrow P(k + 1)$ is true for all positive integers k.
\end{enumerate}

\subsection{Recursive Algorithms}
\begin{enumerate}
\item \underline{Recursive} if it solves a problem by reducing it to an instance of the same problem with smaller input.
\end{enumerate}



\section{Counting}
\subsection{The Pigeonhole Principle}
\begin{enumerate}
\item \underline{The Pigeonhole Principle} -If k is a positive integer and k + 1 or more objects are placed into k boxes, then there is at least one box containing two or more of the objects.
\end{enumerate}

\subsection{Generalized Permutations and Combinations}

\begin{enumerate}
\item \underline{Permutation Without Repetition} - $P( n, r ) = \frac{n!}{(n-r)!}$
\item \underline{Combination Without Repetition } - ${ n \choose r } = \frac{n!}{(n - r)! \times r!}$
\item \underline{Permutation With Repetition} - $P( n, r ) = n^{r}$
\item \underline{Combination With Repetition } - ${ n \choose r } = \frac{(n -r + 1)!}{r!(n + 1)!}$
\end{enumerate}

\section{Relation}
\subsection{Relations}

\begin{enumerate}
\item \underline{Binary Relation} from A to B is a subset of A $\times$ B.
\item \underline{Relation} on a set A is a relation from A to A.
\item \underline{Reflexive} if $(a, a) \in R$ for every element $a \in A.$
\item \underline{Symmetric} if $(b, a) \in R$ whenever $(a, b) \in R$, for all $a, b \in A$
\item \underline{Antisymmetric} A relation R on a set A such that for all $a, b \in A$, if $(a, b) \in R$ and $(b, a) \in R$, then $a = b$
\item \underline{Transitive} if whenever $(a, b) \in R$ and $(b, c) \in R$, then $(a, c) \in R$, for all $a,b,c \in A$.
\item \underline{equivalence relation} For a relation to be equivalent it must be reflexive, symmetric, and transitive.
\item \underline{Partial Order} If it is reflexive, antisymmetric, and transitive
\end{enumerate}

\end{document}
