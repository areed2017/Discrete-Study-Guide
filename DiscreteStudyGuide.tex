\documentclass[12pt]{article}

\usepackage{answers}
\usepackage{setspace}
\usepackage{graphicx}
\usepackage{enumitem}
\usepackage{multicol}
\usepackage{mathrsfs}
\usepackage[margin=1in]{geometry} 
\usepackage{makecell}
\usepackage{amsmath,amsthm,amssymb}
 
\newcommand{\N}{\mathbb{N}}
\newcommand{\Z}{\mathbb{Z}}
\newcommand{\C}{\mathbb{C}}
\newcommand{\R}{\mathbb{R}}

\DeclareMathOperator{\sech}{sech}
\DeclareMathOperator{\csch}{csch}
 
\newenvironment{theorem}[2][Theorem]{\begin{trivlist}
\item[\hskip \labelsep {\bfseries #1}\hskip \labelsep {\bfseries #2.}]}{\end{trivlist}}
\newenvironment{definition}[2][Definition]{\begin{trivlist}
\item[\hskip \labelsep {\bfseries #1}\hskip \labelsep {\bfseries #2.}]}{\end{trivlist}}
\newenvironment{proposition}[2][Proposition]{\begin{trivlist}
\item[\hskip \labelsep {\bfseries #1}\hskip \labelsep {\bfseries #2.}]}{\end{trivlist}}
\newenvironment{lemma}[2][Lemma]{\begin{trivlist}
\item[\hskip \labelsep {\bfseries #1}\hskip \labelsep {\bfseries #2.}]}{\end{trivlist}}
\newenvironment{exercise}[2][Exercise]{\begin{trivlist}
\item[\hskip \labelsep {\bfseries #1}\hskip \labelsep {\bfseries #2.}]}{\end{trivlist}}
\newenvironment{solution}[2][Solution]{\begin{trivlist}
\item[\hskip \labelsep {\bfseries #1}]}{\end{trivlist}}
\newenvironment{problem}[2][Problem]{\begin{trivlist}
\item[\hskip \labelsep {\bfseries #1}\hskip \labelsep {\bfseries #2.}]}{\end{trivlist}}
\newenvironment{question}[2][Question]{\begin{trivlist}
\item[\hskip \labelsep {\bfseries #1}\hskip \labelsep {\bfseries #2.}]}{\end{trivlist}}
\newenvironment{corollary}[2][Corollary]{\begin{trivlist}
\item[\hskip \labelsep {\bfseries #1}\hskip \labelsep {\bfseries #2.}]}{\end{trivlist}}
 
\begin{document}
 
% --------------------------------------------------------------
%                         Start here
% --------------------------------------------------------------
 
\title{FINAL Study Guide}%replace with the appropriate homework number
\author{Andrew Reed\\ %replace with your name
Discrete Final} %if necessary, replace with your course title
 
\maketitle

\tableofcontents

\pagebreak

\quad

\quad

\section{The Foundations: Logic and Proofs}

\subsection{Propositional Logic}

Examples of a proposition:

$p(x) = x$ is a cat.

$q(x) = x$ has fur.

\begin{enumerate}
\item \underline{Negation}, $\neg p(x)$, changing the statement to $x$ is not a cat
\item \underline{Conjunction}, $p(x) \wedge q(x)$, changing the statement to say $x$ is a cat and it has fur.
\item \underline{Disjunction}, $p(x) \lor q(x)$, changing the statement to say $x$ is a cat and it does not have fur.
\item \underline{Exclusive Or},  $p(x) \oplus q(x)$, where the statement is true only when exactly one of $p(x)$ or $q(x)$ is true, otherwise the statement is false.
\item \underline{Conditional Statement}, $p(x) \rightarrow q(x)$,  where the statement is read "If p, then q" the statement is false when p is true and q is false, otherwise the statement is true.
\item \underline{Bi-Conditional Statement}, $p(x) \leftrightarrow q(x)$, where the statement is true if the values of p and q match. The statement is read as any of the following;
	\begin{enumerate}
	\item "p if and only if q"
	\item "p is necessary and sufficient for q"
	\item "if p then q, and conversely"
	\item "p iff q"
	\end{enumerate}
\end{enumerate}

Definitions
\begin{enumerate}
\item \underline{Hypothesis} - p is considered the hypothesis in the statement $p(x) \rightarrow q(x)$.
\item \underline{Conclusion}- q is considered the conclusion in the statement $p(x) \rightarrow q(x)$.
\item \underline{Bi-Implications} - Another way to express "Bi-Conditional statements"
\end{enumerate}

\quad

\quad

\subsection{Propositional Equivalences}

\begin{enumerate}
\item \underline{Logically Equivalent}, $p(x) \equiv q(x)$, when two statements share the same truth values then the statement is said to be logically equivalent.
\item \underline{Identity Laws} 
\[
	p(x) \wedge T \equiv p(x)
\]
\[
	p(x) \lor F \equiv p(x)
\]

\item \underline{Domination Laws}
\[
	p(x) \lor T \equiv T 
\]
\[
	p(x) \wedge F \equiv F
\]

\item \underline{Idempotent Laws}
\[
	p(x) \lor p(x) \equiv p(x)
\]
\[
	p(x) \wedge p(x) \equiv p(x)
\]

\item \underline{Double Negation Law}
\[
	\neg (\neg p(x)) \equiv p(x)
\]

\item \underline{Commutative Laws}
\[
	p(x) \lor q(x) \equiv q(x) \lor p(x) 
\]
\[
	p(x) \wedge q(x) \equiv q(x) \wedge p(x)
\]

\item \underline{Associative Laws}
\[
	\left ( p(x) \lor q(x) \right ) \lor r(x) \equiv  p(x) \lor \left ( q(x)  \lor r(x) \right )
\]
\[
	\left ( p(x) \wedge q(x) \right ) \wedge r(x) \equiv  p(x) \wedge \left ( q(x)  \wedge r(x) \right )
\]

\item \underline{Distributive Laws}
\[
	p(x) \lor \left ( q(x)  \wedge r(x) \right ) \equiv \left ( p(x) \lor q(x) \right ) \wedge \left ( p(x) \lor r(x) \right )
\]
\[
	p(x) \wedge \left ( q(x)  \lor r(x) \right ) \equiv \left ( p(x) \wedge q(x) \right ) \lor \left ( p(x) \wedge r(x) \right )
\]

\item \underline{De Morgan's Laws}
\[
	\neg \left ( p(x) \wedge q(x) \right ) \equiv \neg p(x) \lor \neg q(x)
\]
\[
	\neg \left ( p(x) \lor q(x) \right ) \equiv \neg p(x) \wedge \neg q(x)
\]

\item \underline{Absorption Laws}
\[
	p(x) \lor \left ( p(x) \wedge q(x) \right ) \equiv p(x)
\]
\[
	p(x) \wedge \left ( p(x) \lor q(x) \right ) \equiv p(x)
\]

\item \underline{Negation Laws}
\[
	p(x) \lor \neg p(x) \equiv T
\]
\[
	p(x) \wedge \neg p(x) \equiv F
\]

\end{enumerate}

Definitions
\begin{enumerate}
\item \underline{Tautology} - When all cases in the statement are determined to be true, it is said to be a tautology.
\item \underline{Contradiction}- When all cases in the statement are determined to be false, it is said to be a contradiction.
\item \underline{Contingency} - When the statement is neither a tautology, nor a contradiction then it is said to be a contingency.
\end{enumerate}

\quad

\quad

\subsection{Predicates and Quantifiers}

\quad 

\begin{enumerate}
\item \underline{Universal Quantification}, $\forall p(x)$ read as, "For all $x$, $p(x)$."
\item \underline{Existential Quantification}, $\exists p(x)$ read as, "There exists an element $x$, that $p(x)$."
\item \underline{De Morgan's Laws for Quantifiers}:
\[
	\neg \forall p(x) \equiv \exists \neg p(x)
\]
\[
	\neg \exists p(x) \equiv \forall \neg p(x)
\]
\end{enumerate}

Definitions
\begin{enumerate}
\item \underline{Quantification} - Used to express the extent that a predicate is true. In English, the words; all, some, many, none, and few are used in quantifications.
\item \underline{Counterexample} - When there is a value, $x$ in which $\forall p(x)$ is false, then that value of $x$ is called a counterexample of $\forall p(x)$.
\end{enumerate}

\pagebreak

\subsection{Rules of Inference (ROI)}

\quad 

\begin{center}
\begin{tabular}{ c c }

\makecell{ \underline{Modus Ponens} \\ p \\ p $\rightarrow$ q \\ $\overline{\therefore{q}}$ \\ \quad}

&

\makecell{ \underline{Modus Tollens} \\ $\neg q$ \\ p $\rightarrow$ q \\ $\overline{\therefore{\neg p}}$ \\ \quad}

\\

\makecell{ \underline{Hypothetical Syllogism} \\ p $\rightarrow$ q \\ q $\rightarrow$ r \\ $\overline{\therefore{p \rightarrow r}}$ \\ \quad}

&

\makecell{ \underline{Disjunctive Syllogism} \\ p $\lor$ q \\ $\neg$ p \\ $\overline{\therefore{q}}$ \\ \quad}

\\

\makecell{ \underline{Addition} \\ p \\ $\overline{\therefore{p \lor q}}$ \\ \quad}

&

\makecell{ \underline{Simplification} \\ p $\wedge$ q \\ $\overline{\therefore{p}}$ \\ \quad}

\\

\makecell{ \underline{Conjunction} \\ p \\ q \\ $\overline{\therefore{p \wedge q}}$ \\ \quad }

&

\makecell{ \underline{Universal Instantiation} \\ $\forall p(x)$ \\ $\overline{\therefore{p(c)}}$ \\ \quad }

\\

\makecell{ \underline{Universal Generalization} \\ $p(c)$ for an arbitrary c \\ $\overline{\therefore{\forall p(x)}}$ \\ \quad}

&

\makecell{ \underline{Existential Instantiation} \\ $\exists p(x)$ \\ $\overline{\therefore{p(c)}}$ for some element c \\ \quad}

   
\end{tabular}

\quad

\quad

\underline{Existential Generalization}

$p(c)$ for some element c

 $\overline{\therefore{\exists p(x)}}$ 

\end{center}


Definitions
\begin{enumerate}
\item \underline{Argument} - Sequence of propositions.
\item \underline{Premises} - All propositions in the argument with the exclusion of the conclusion
\item \underline{Conclusion} - The final proposition in the argument.
\item \underline{Argument Form} - A sequence of compound propositions involving propositional variables.
\item \underline{Valid} - A form that makes it impossible for the premises to be true and the conclusion nevertheless to be false.
\end{enumerate}

\end{document}
