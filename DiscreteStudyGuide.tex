\documentclass[12pt]{article}

\usepackage{answers}
\usepackage{setspace}
\usepackage{graphicx}
\usepackage{enumitem}
\usepackage{multicol}
\usepackage{mathrsfs}
\usepackage[margin=1in]{geometry} 
\usepackage{amsmath,amsthm,amssymb}
 
\newcommand{\N}{\mathbb{N}}
\newcommand{\Z}{\mathbb{Z}}
\newcommand{\C}{\mathbb{C}}
\newcommand{\R}{\mathbb{R}}

\DeclareMathOperator{\sech}{sech}
\DeclareMathOperator{\csch}{csch}
 
\newenvironment{theorem}[2][Theorem]{\begin{trivlist}
\item[\hskip \labelsep {\bfseries #1}\hskip \labelsep {\bfseries #2.}]}{\end{trivlist}}
\newenvironment{definition}[2][Definition]{\begin{trivlist}
\item[\hskip \labelsep {\bfseries #1}\hskip \labelsep {\bfseries #2.}]}{\end{trivlist}}
\newenvironment{proposition}[2][Proposition]{\begin{trivlist}
\item[\hskip \labelsep {\bfseries #1}\hskip \labelsep {\bfseries #2.}]}{\end{trivlist}}
\newenvironment{lemma}[2][Lemma]{\begin{trivlist}
\item[\hskip \labelsep {\bfseries #1}\hskip \labelsep {\bfseries #2.}]}{\end{trivlist}}
\newenvironment{exercise}[2][Exercise]{\begin{trivlist}
\item[\hskip \labelsep {\bfseries #1}\hskip \labelsep {\bfseries #2.}]}{\end{trivlist}}
\newenvironment{solution}[2][Solution]{\begin{trivlist}
\item[\hskip \labelsep {\bfseries #1}]}{\end{trivlist}}
\newenvironment{problem}[2][Problem]{\begin{trivlist}
\item[\hskip \labelsep {\bfseries #1}\hskip \labelsep {\bfseries #2.}]}{\end{trivlist}}
\newenvironment{question}[2][Question]{\begin{trivlist}
\item[\hskip \labelsep {\bfseries #1}\hskip \labelsep {\bfseries #2.}]}{\end{trivlist}}
\newenvironment{corollary}[2][Corollary]{\begin{trivlist}
\item[\hskip \labelsep {\bfseries #1}\hskip \labelsep {\bfseries #2.}]}{\end{trivlist}}
 
\begin{document}
 
% --------------------------------------------------------------
%                         Start here
% --------------------------------------------------------------
 
\title{FINAL Study Guide}%replace with the appropriate homework number
\author{Andrew Reed\\ %replace with your name
Discrete Final} %if necessary, replace with your course title
 
\maketitle

\tableofcontents

\pagebreak

\section{The Foundations: Logic and Proofs}

\subsection{Propositional Logic}

Examples of a proposition:

$p(x) = x$ is a cat.
$q(x) = x$ has fur.

\begin{enumerate}
\item \underline{Negation}, $\neg p(x)$, changing the statement to $x$ is not a cat
\item \underline{Conjunction}, $p(x) \wedge q(x)$, changing the statement to say $x$ is a cat and it has fur.
\item \underline{Disjunction}, $p(x) \lor q(x)$, changing the statement to say $x$ is a cat and it does not have fur.
\item \underline{Exclusive Or},  $p(x) \oplus q(x)$, where the statement is true only when exactly one of $p(x)$ or $q(x)$ is true, otherwise the statement is false.
\item \underline{Conditional Statement}
\end{enumerate}

\subsection{Trigonometry Derivatives}

\begin{enumerate}
\item 
\end{enumerate}

\subsection{Inverse Trigonometry Derivatives}

\pagebreak



\end{document}
